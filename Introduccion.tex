\documentclass[11pt,a4paper]{report}
\usepackage[utf8]{inputenc}
\usepackage[spanish]{babel}
\usepackage{amsmath}
\usepackage{amsfonts}
\usepackage{amssymb}
\usepackage{makeidx}
\usepackage{graphicx}
\usepackage[left=1cm,top=1cm,right=1cm,bottom=1cm]{geometry} 
\title{Introduccion}
\begin{document}
	\begin{center}
		\begin{huge}
		Introducci\'on.\\
		\vspace*{1.5cm}
		\end{huge}
		\end{center}
	
\begin{flushleft}
	
		El proposito de esta pr\'actica es aprender a realizar mediciones de resistencias por medio del mult\'imetro e interpretar el valor de las resistencias por medio de sus codigos de colores.\\
		\vspace*{0.5cm}
		
		Primeramente definiremos algunos conceptos.\\
		\vspace*{0.5cm}
		\begin{Large}
				Mult\'imetro.\\
		\end{Large}
		Es un instrumento de medida que ofrece la posibilidad de medir distintos parametros electricos y magnitudes en el mismo aparato. Las más comunes son las de voltímetro, amperímetro y óhmetro.\\
		\vspace*{0.5cm}
		Las diferentes funciones comunes en multimetros actuales son:\\
		Medición de resistencia\\
		Prueba de continuidad\\
		Medición de tensiones de CA y CC\\
		Medición de milivoltios de CA y CC\\
		Medición de corriente alterna y continua\\
		Medición de capacitancia (algunos modelos)\\
		Medición de frecuencia (algunos modelos)\\
		\vspace*{0.5cm}
		
		\begin{Large}
			Resistencia.\\
		\end{Large}
	
	 Fuerza que rechaza o se opone a los electrones que se desplazan en algún material. La resistencia eléctrica se mide en Ohm. Una de las propiedades físicas de los materiales es su resistencia física a la electricidad. Según su resistencia se dividen en dos tipos:\\
	 \vspace*{0.5cm}
	 		
	 \textbf{Aislantes:} son materiales con gran resistencia eléctrica como lo son, por	 \vspace*{0.5cm} ejemplo, el plástico y la cerámica.\\ 	
	 \vspace*{0.5cm}
	 \textbf{Conductores:} permiten el libre flujo de los electrones debido a su baja resistencia eléctrica. Los metales, en general, son grandes conductores.\\
	 \vspace*{0.5cm}
	 La resistencia eléctrica varía dependiendo de otras características físicas del producto como:\\
	 
	 \textbf{El grosor:}: mientras más grueso el conductor menor es la resistencia.\\
	 \textbf{La largura:} mientras más largo, mayor es la resistencia.\\
	 \textbf{La conductividad:} mientras menor es la resistividad, mayor será la conductividad.\\
	 \textbf{La temperatura:} a mayor temperatura, mayor será la resistencia.\\
	 \vspace*{0.5cm}
	 
	 \begin{Large}
	 	C\'odigo de colores de las Resistencias.\\
	 	 \vspace*{0.5cm}
	 \end{Large}
 
 	El c\'odigo de colores de las resistencias nos permiten conocer el valor en ohms que son capaces de soportar. En la mayoria de los casos las resistencias presentan 4 bandas de colores, cuando es asi se interpretan de la siguiente forma:\\
 	\vspace*{0.5cm}
 	
 	\textbf{Bandas 1 y 2:}: representan los dos primeros valores de la resistencia.\\
 	\textbf{Banda 3:} representa el factor multiplicador de la resistencia, es decir, los valores de las primeras dos resistencias se multiplicaran por el factor multiplicador de la tercer banda.\\
 	\textbf{Banda 4:} representa el valor de la tolerancia de la resistencia, es decir, el valor obtenido de la resistencia puede varias + o - el valor encontrado en la cuarta banda respecto al valor obtenido en las 3 primeras bandas.\\
 	\vspace*{0.5cm}
 	
 	Cuando la resistencia presenta una quinta banda, para obtener el valor de dicha resistencia se utiliza el procedimiento mencionado anteriormente pero la tercer banda tambien representara un numero, la cuarta banda representara al factor multiplicador y la quinta banda representara la tolerancia de la resistencia.\\
 	
			
\end{flushleft}

	
\end{document}