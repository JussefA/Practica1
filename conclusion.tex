\documentclass[12pt,a4paper]{report}
\usepackage[utf8]{inputenc}
\usepackage[spanish]{babel}
\usepackage{amsmath}
\usepackage{amsfonts}
\usepackage{amssymb}
\usepackage{makeidx}
\usepackage{graphicx}
\usepackage[left=1.00cm, right=1.00cm, top=1.00cm, bottom=1.00cm]{geometry}
\title{Conclusion}
\begin{document}
	\begin{center}
		\begin{Large}
			Conclusi\'on.\\
			\vspace{1.5cm}
		\end{Large}
	\end{center}
\begin{flushleft}
	Al finalizar esta práctica se comprendió que los colores que contenían las 5 diferentes resistencias dadas por el docente tenían valores diferentes puesto que estos comprendían de diferentes colores, algunas de ellas comprendían de los mismos colores, pero variaban en la terminación dorada o plateada, pero esto no significaba que tenían el mismo valor.\\
	\vspace*{0.5cm}
	Con la ayuda de una tabla la cual contenía los valores de cada color presentado en la resistencia y algunos apuntes, el equipo se dio a la tarea de obtener los valores de las primeras líneas de colores y finalizando con una multiplicación de la penúltima línea del color designado en dicha resistencia. Después de obtener dicho valor se buscaba el valor exacto o cercano a este en el multímetro utilizado.\\
	\vspace*{0.5cm}
	Al finalizar la medición de cada una de las resistencias nos dimos cuenta que algunas de ellas sus valores no eran exactos ya fuese la medición o nuestro calculo, pero de ahí en fuera todo tenía el mismo valor calculado.
\end{flushleft}
	
\end{document}